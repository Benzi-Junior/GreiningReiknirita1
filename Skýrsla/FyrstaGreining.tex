Fyrst prófuðum við að keyra grófa tíma og minnismælingu bæði fyrir slembifylki og svo nokkur valin sértilfelli  á öllum föllunum.
Keyrslan var fyrst gerð með fylkjum af fastri stærð og lítið hægt að lesa úr gögnunum annað en hvaða föll notuðu minni á kös og hver ekki þar sem ekki er hægt að mæla stærð hlaðans í java með þægilegum hætti.
Út úr  því fengum við að föll s2, s3, s4, s7 og s10 notuðu ekki minni á kös og svo þegar forritinu var breytt til að fá mælingar á mis stórum fylkjum kom í ljós að mynnið sem s1 notaði var fasti, það er mynnistnotkunin var óháð stærð fylkisins.

Annað sem kom fram í þessari fyrstu mælingu að sum fallana kláruðu minnið á hlaðanum fyrir frekar lítil slembifylki (undir $10^5$ stök)  en það voru föll s3, s4, s7 og  s10.
Af þeim sem aðferðum fast minni eða ekkert á kös eru bara flýtiröðun sem notar ekki fast minni því ályktuðum við að það væru s3, s4, s7 og  s10 sem væru flýtiröðun og þá væru s1 og s2 sem væru innsetningarröðun og hrúguröðun.


