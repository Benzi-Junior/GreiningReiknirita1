Til að greina á milli mismunandi leiða til að velja vendistak í flýtiröðun byrjuðum við á að skoða hvert þeirra kláraði minnið á hlaðanum fyrir minnsta raðaða fylkið.
Fyrir raðað fylkið þarf ekki að framkvæma neinar skiptingar en það er versta tilfelli aðferðarinnar sem velur alltaf aftasta stak sem vendistak í minnisflækju það er hlaðin mun þá stækka í línulegu hlutfalli við lengd fylkisins 

Með því að auka alltaf stærð fylkisins var það s3 sem fyrst kláraði hlaðan hinar aðferðirnar verandi allar nokkuð skilvirkar  fyrir raðað fylki og kláruðu því ekki hlaðan í prófunum. Svo s3 er flýtiröðun sem notar aftasta stak sem vendista.

Því næst athuguðum við "sagtannar fylki" það er fylki sem er samansett úr röðuðum hlutfylkjum
Ef fjölidi hlutfylkjanna er slétt tala þá mun aðferðirn sem velur miðgildi fremsta aftasta og miðstaksins og aðferðin sem velur  alltaf miðjustakið, hafa stærsta eða minnsta gildi sem vendistak og eru því mjög óhentugar fyrir það tilfelli, aftur á móti ef fylkið er raðað þá munu þær tvær aðferðir ávalt velja besta vendistak og klára sem hraðast en aðferðin sem velur slembistak verður óhentugri.
Þannig að með því að prófa á þeim þremur flýtiröðunaraðferðum sem eftir eru hraðan við að raða sagtannarfylki og röðuðu fylki fékkst að s4 og s7 tóku nokkurn vegin sama tíma að jafnaði í báðum tilfellum en s10 tók mun skemmri tíma við að raða sagtannar fylkinu en mun lengri til að raða raðaða  fylkinu (í báðum tilfellum tóku hægari aðferðirnar um tvöfallt lengri tími)

Svo s10 er flýtiröðun með sem velur vendistak af handahófi. 
